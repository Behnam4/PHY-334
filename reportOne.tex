\documentclass[fleqn]{article}
\oddsidemargin 0.0in
\textwidth 6.0in
\thispagestyle{empty}
\usepackage{import}
\usepackage{amsmath}
\usepackage{graphicx}
\usepackage{flexisym}
\usepackage{calligra}
\usepackage{amssymb}
\usepackage{bigints} 
\usepackage[english]{babel}
\usepackage[utf8x]{inputenc}
\usepackage{float}
\usepackage[colorinlistoftodos]{todonotes}


\DeclareMathAlphabet{\mathcalligra}{T1}{calligra}{m}{n}
\DeclareFontShape{T1}{calligra}{m}{n}{<->s*[2.2]callig15}{}
\newcommand{\scriptr}{\mathcalligra{r}\,}
\newcommand{\boldscriptr}{\pmb{\mathcalligra{r}}\,}

\definecolor{hwColor}{HTML}{AD53BA}

\begin{document}

  \begin{titlepage}

    \newcommand{\HRule}{\rule{\linewidth}{0.5mm}}

    \center

    \begin{center}
      \includegraphics[height=11cm, width=11cm]{asu.png}
    \end{center}

    \vline

    \textsc{\LARGE Advanced Laboratory I}\\[1.5cm]

    \HRule \\[0.5cm]
    { \huge \bfseries Compton Scattering}\\[0.4cm] 
    \HRule \\[1.0cm]

    \textbf{Behnam Amiri}

    \bigbreak

    \textbf{Prof: Ralph Chamberlin}

    \bigbreak

    \textbf{Lab Partners: Daniel Henningsen, Micah Smith, Srihari Ravi}

    \bigbreak

    \textbf{{\large \today}\\[2cm]}

    \vfill

  \end{titlepage}

  \textbf{Abstract}

  \vspace{10px}

  The Compton scattering or in other words, inelastic scattering of electrons by X-rays is a powerful probe to investigate the behavior of valence 
  electrons in any material. Spectra of Compton scattered photons provides peculiar information about the electron momentum distribution of 
  valence electrons of target materials and hence their electronic properties. If the incident radiation has degree of circular polarization, 
  then the Compton scattering from unpaired electrons, so called magnetic Compton scattering, enables to probe the spin momentum 
  distribution in the ferro- and ferri-magnetic materials.

  \vspace{20px}

  \textbf{I. Introduction}

  \vspace{10px}

  By $1920$ the successes of the quantum theories of blackbody spectra (Planck, $1901$), the photoelectric effect (Einstein, $1905$) 
  and the hydrogen spectrum (Bohr, $1913$) had established the idea that interactions between electromagnetic radiation of frequency $v$ and matter
  occuer through the emission or absorption of discrete quanta of energy $E=hv$. The next crucial step in the development of the modern concept
  of the photon as the particle of electromagnetic radiation was taken by Arthur Compton in the interpretation 
  of experiments he initiated in $1920$ to measure with precision the wavelengths of X-rays scattered from electrons in materials of 
  (low atomic number). The phenomena of X-ray scattering had already been studied intensively. It was known that the 
  penetrating power of X-rays decreases with increasing wavelength and that X-rays are less penetrating after being scattered than 
  before, which indicated that the scattering process somehow increases their wavelength. 

  Compton’s idea was to use the recently developed 
  technique of high-resolution X-ray spectrometry, based on measurement of the angle of Bragg reflection of X-rays from crystals, 
  to measure precisely the wavelengths of the scattered X-rays. Irradiating a carbon target with an intense collimate beam of monochromatic molybdenum 
  $K_{\alpha}$ X-rays and using an ionization chamber as the detector in his spectrometer, Compton found that the
  spectrum of scattered X-rays had two distinct spectral lines, one at the wavelength of the incident X-rays and another at a wavelength that was 
  longer by an amount that depends on the angle of scattering. The scattering without a wavelength shift was readily explained by the classical theory 
  of coherent scattering of electromagnetic waves from electrons bound in atoms. However, the classical theory provided no explanation of the 
  wavelength-shifting \emph{incoherent} scattering process. The phenomenon of Bragg reflection used in Compton’s measurements was a clear demonstration 
  of the wavelike character of the X-rays. 

  Nevertheless, Compton put forward the apparently contradictory idea that X-rays, known to be electromagnetic radiation of very short wavelength,
  interact with electrons like particles of zero rest mass so that their energy $E=hv=h \dfrac{c}{\lambda}$ and momentum $p$ are related by the
  relativistic equation for particles of zero rest mass, namely $p=\dfrac{E}{c}$. He calculated according to relativistic mechanics the relations between the initial
  and final energies and momenta of an X-ray quantum and a free electron involved, like billiard balls, in an elastic collision. In this way he arrived at the formula for 
  the \emph{Compton shift} in the wavelength of incoherently scattered X-rays
  $$\Delta \lambda=\dfrac{h}{mc} \left(1-cos(\theta)\right)$$


  \bigbreak

  \textbf{II. Background Information}

  \textbf{III. Theory}

  \textbf{IV. Experimental Procedure}

  \textbf{V. Results}

  \textbf{VI. Discussion}

  \textbf{VII. Conclusions}

  \textbf{VIII. References}

\end{document}